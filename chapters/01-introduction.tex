\section{INTRODUCTION}
	\paragraph{}
		Herein is contained the \textit{Liber Defensionis}, a fortress of spirit constructed from three ancient foundations: the \textit{Ars Goetia} (The Lesser Key of Solomon), the Babylonian protective rites of the \textit{Udug-\d{h}ul}, and the exorcistic authority of the \textit{Rituale Romanum}.

		While these traditions span millennia and vast theological divides, they share a single, immutable principle: the invocation of Supreme Divine Authority to compel the obedience of lesser spiritual entities.

		The Goetic system (\cite{mathersgoetia1904}) is famously known for the conjuration of spirits, yet hidden within its circles and triangles lies a perfect defensive architecture. The magician was never intended to stand naked before the abyss; the system provides the names of compulsion, the seals of binding, and the geometry of containment. This work inverts the summoner's logic: rather than opening a door to call a spirit \textit{in}, we utilize the keys to lock the spirit \textit{out}.

		Complementing this geometry is the primal authority of the Babylonian tradition. Drawing from the \textit{En\={u}ma Eli\v{s}}, we utilize the Fifty Names of Marduk. In the ancient \textit{Marduk’s Address to the Demons}, the \textit{\={a}\v{s}ipu} (exorcist) did not merely pray to the god—he \textit{became} the god, speaking with the voice of the one who slew the chaos-dragon Ti\={a}mat.

		Finally, we anchor these ancient forms in the spoken authority of the Roman Catholic tradition, utilizing the \textit{Vade Retro Satana} and the \textit{Prayer to St. Michael}, formulas tempered by centuries of liturgical use against spiritual oppression.

		\paragraph{A Note on Illustrations}
		The geometric figures presented in this work follow the rectified schematics of \textcite{mathersgoetia1904} for the sake of operational clarity. The reader should note that the original 17\textsuperscript{th} century manuscripts, as critically examined by \textcite{peterson2001}, depict these seals in significantly more fluid, hand-drawn styles. We rely on the Victorian geometric precision here strictly for the construction of stable defensive barriers.\bigskip

		\noindent \textbf{The Structure of the Work:}

		\begin{description}
			\item[PART I: Repulsion] \hfill \\
			The establishment of barriers to prevent entry.
			\item[PART II: Entrapment] \hfill \\
			The geometry of containment for localized disturbances.
			\item[PART III: Expulsion] \hfill \\
			The rites of driving out that which has already embedded itself.
			\item[PART IV: The Fifty Names of Marduk] \hfill \\
			The assumption of Babylonian divine authority.
			\item[PART V: Roman Catholic Formulas] \hfill \\
			The traditional litanies of Christian exorcism.
		\end{description}
\clearpage

