	\section{THE FIFTY NAMES OF MARDUK}
	\hspace{1cm}{\large Babylonian Divine Authority for Exorcism}
	
	\subsection{Origin and Function}
	The Fifty Names of Marduk derive from the Babylonian creation epic, the {Enuma Elish}{\index{Enuma Elish}} (circa 12\textsuperscript{th} century BCE). These names appear in Tablets VI and VII and represent powers given to Marduk after his victory over the chaos dragon Ti\={a}mat.
	
	These names were used in \enquote{Marduk's Address to the Demons,{\textquotedblright} a key section of the ancient Mesopotamian exorcism series \index{Udug-\d{h}ul}Udug-\d{h}ul ({\textquotedblleft}Evil Demons}). When performing exorcisms, the exorcist (\=a\v{s}ipu) would essentially become Marduk, speaking in the first person as Asallu\d{h}i/Marduk.
	
	The core principle: By reciting the Fifty Names, the exorcist assumes the identity and authority of Marduk himself---the god who defeated primordial chaos.
	
\subsection{The Fifty Names}
\begin{xltabular}{\textwidth}{|c|l|X|}
	\hline
	\textbf{\#} & \textbf{Name} & \textbf{Domain/Power} \\ \hline
	\endfirsthead
	\hline
	\textbf{\#} & \textbf{Name} & \textbf{Domain/Power} \\ \hline
	\endhead
	
	1 & \textsc{Marduk} & The Son of the Sun; he who establishes cosmic order. \\ \hline
	2 & \textsc{Marukka} & He who creates the wisdom of all. \\ \hline
	3 & \textsc{Marutukku} & Master of the arts of protection; hostile to the aggressor. \\ \hline
	4 & \textsc{Barashaku\v{s}u} & He who compelled the obedient into chaos; he who is wide of heart. \\ \hline
	5 & \textsc{Lugaldimmerankia} & Master of Order, who puts the wind demons to flight. \\ \hline
	6 & \textsc{Nari} & The Director, who regulates the heavens and earth. \\ \hline
	7 & \textsc{Asallu\d{h}i} & The Name of Power; the protector of the good, the banisher of the wicked. \\ \hline
	8 & \textsc{Namtila} & The Life-Giver; he who restores the dead to life. \\ \hline
	9 & \textsc{Namru} & The Bright One; he who purifies the path. \\ \hline
	10 & \textsc{Asaru} & Bestower of cultivation; who establishes the water levels. \\ \hline
	11 & \textsc{Asarualim} & He whose counsel is esteemed; the light of the gods. \\ \hline
	12 & \textsc{Asarualimnunna} & The Power that presides over the armor of the gods. \\ \hline
	13 & \textsc{Tutu} & He who silences the weeping; the supreme banisher. \\ \hline
	14 & \textsc{Ziukkina} & Life of the Host; he who establishes the heavens. \\ \hline
	15 & \textsc{Ziku} & The God of the Breath; Lord of Holiness. \\ \hline
	16 & \textsc{Agaku} & The Lord of the Charm; he who brings the dead to life. \\ \hline
	17 & \textsc{Tuku} & The Lord of the Ban; he whose spell is holy. \\ \hline
	18 & \textsc{\v{S}azu} & He who knows the heart; the searcher of souls. \\ \hline
	19 & \textsc{Zisi} & The Reconciler; the silencer of the enemy. \\ \hline
	20 & \textsc{Su\d{h}rim} & He who roots out the enemies with the weapon. \\ \hline
	21 & \textsc{Su\d{h}gurim} & He who destroys the foes; who makes the shell to vanish. \\ \hline
	22 & \textsc{Za\d{h}rim} & The Destroyer; he who grinds the enemy to dust. \\ \hline
	23 & \textsc{Za\d{h}gurim} & He who destroys the enemy in a specific way (the specialized destroyer). \\ \hline
	24 & \textsc{Enbilulu} & The Lord of the Abundance; he who governs the waters. \\ \hline
	25 & \textsc{Epadun} & The Lord of the Plain; who irrigates the fields. \\ \hline
	26 & \textsc{Gugal} & The Inspector of the Canals; the Lord of the Overflow. \\ \hline
	27 & \textsc{Hegal} & He who accumulates the harvest; who brings rain. \\ \hline
	28 & \textsc{Sirsir} & He who heaps up the mountain (of grain) upon the serpent. \\ \hline
	29 & \textsc{Mala\d{h}} & The Boatman; he who navigates the ship of state. \\ \hline
	30 & \textsc{Gil} & The Storehouse; the seed-preserver. \\ \hline
	31 & \textsc{Gilma} & The Founder; who renders the firmament permanent. \\ \hline
	32 & \textsc{Agilma} & The Lofty One; who removes the hoop (of constraint). \\ \hline
	33 & \textsc{Zulum} & The Divider; who assigns the fields. \\ \hline
	34 & \textsc{Mummu} & The Creator of the Universe; the One who was before. \\ \hline
	35 & \textsc{Zulummar} & The Hammer; he who crushes the opposition. \\ \hline
	36 & \textsc{Lugalabdubur} & The Destroyer of the Gods of Ti\={a}mat; the conqueror. \\ \hline
	37 & \textsc{Pagalguenna} & The Leader of all Lords; whose spirit is pre-eminent. \\ \hline
	38 & \textsc{Lugaldurma\d{h}} & The King of the Bond of the Gods; the Lord of the Link. \\ \hline
	39 & \textsc{Aranunna} & The Counselor of Ea; the creator of the treaty. \\ \hline
	40 & \textsc{Dumuduku} & The Son of the Holy Mound; the place of renewal. \\ \hline
	41 & \textsc{Lugalanna} & The King of the Height; the Lord of the Strength. \\ \hline
	42 & \textsc{Lugalugga} & The Lord of the Hosts of the Dead; he who extracts the life. \\ \hline
	43 & \textsc{Irkingu} & He who carried off the Kingu (Dragon-General) in the battle. \\ \hline
	44 & \textsc{Kinma} & The Judge and Governor of the Gods; the director of names. \\ \hline
	45 & \textsc{Esizkur} & He who sits in the House of Prayer; the hearer of vows. \\ \hline
	46 & \textsc{Gibi} & The One who maintains the Point; the creator of the winds. \\ \hline
	47 & \textsc{Addu} & The Storm King; who covers the sky with his shroud. \\ \hline
	48 & \textsc{A\v{s}aru} & The Overseer; who regulates the gods of destiny. \\ \hline
	49 & \textsc{Nebiru} & The Star of the Pole; he who holds the pivot of the stars. \\ \hline
	50 & \textsc{Nibbura} & Father Enlil bestowed this name: The Lord of the Lands. \\ \hline
	\caption[The Fifty Names of Marduk]{The Fifty Names of Marduk \\ \cite{speiser1969}}
	\end{xltabular}
	
	\subsection{Using the Names in Exorcism}
	The exorcist declares each name in the first person, assuming Marduk's identity:
	
	\enquote{I am Asallu\d{h}i, magician of the gods.  I am MARDUK, supreme authority.  I am ASARULUDU, wielder of the flaming sword.  I am NAM\-TIL\-LA\-KU, who speaks with spirits.  I am LUG\-GAL\-DIM\-MER\-ANK\-IA, who commands the wind demons and puts order into chaos.  By these names, I command thee: DEPART.}
	
	
	\bigskip
	
	\clearpage

