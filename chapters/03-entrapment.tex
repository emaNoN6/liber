\section{ENTRAPMENT}
\hspace{1cm}{\large Containing What Is Present and Localized}
\subsection{The Logic of Containment}
\paragraph{}{Repulsion prevents entry. Entrapment does the opposite—it creates a space the entity cannot leave. Where the Circle keeps things out, the trap keeps things in.}

The mechanism is attraction, not force. A correctly configured trap presents as a point of least resistance, a space where the entity's nature is accommodated rather than opposed. Once entered, the geometric boundary closes. The entity remains not because it is physically constrained, but because the exit requires a level of authority it does not possess.

This is critical: A trap does not work by overpowering the entity. It works by creating conditions the entity cannot navigate without external release.
\subsection{Entrapment vs. Repulsion: Key Differences}
\begin{xltabular}{\textwidth}{|X|X|}
	\hline
	\textbf{Repulsion} & \textbf{Entrapment} \\ \hline
	\endfirsthead
	\hline
	\textbf{Repulsion} & \textbf{Entrapment} \\ \hline
	\endhead
	Prevents approach & Draws in \\ \hline
	Closed boundary pointing outward & Open boundary that closes after entry \\ \hline
	Divine names assert dominance & Divine names establish inescapability \\ \hline
	Used preventatively & Used after intrusion detected \\ \hline
	Active resistance & Passive containment \\ \hline
	\caption{Repulsion vs. Entrapment}
\end{xltabular}
Understanding this distinction determines which tool to deploy. If the entity is outside and you want it to stay there: repulsion. If the entity is already present and you need to isolate it before expulsion: entrapment.
\subsection{The Triangle of Art: Trap Configuration}
\paragraph{}{The Triangle functions differently when used for entrapment rather than repulsion. The orientation reverses.}

\textbf{Configuration for Entrapment:}

Point the apex INWARD toward the suspected location of the entity. This creates a converging geometry that draws the entity toward the central point.
Inscribe the three sides with the same names as the repulsion configuration:
\begin{itemize}
	\item PRIMEUMATON (commands origin—the entity's beginning is bounded)
	\item ANAPHAXETON (binds the formless—prevents shapeshifting or dissolution)
	\item TETRAGRAMMATON (ultimate authority—no appeal, no escape)
\end{itemize}
Place the Hexagram of Solomon at the center, again without a specific sigil. The blank hexagram functions as a universal holding pattern.
\textbf{Deployment Procedure:}
\begin{enumerate}
	\item Identify the area of disturbance (window, mirror, doorway, object)
	\item Place the triangle over or around the affected area with apex pointing inward
	\item Drive iron nails into the ground at each of the three corners
	\item Place a small pile of salt at each point
	\item Light three black candles at the corners (black absorbs, does not repel)
	\item Speak the constraint formula (see below)
\end{enumerate}
\textbf{Constraint Formula:}
\begin{quotation}
	\enquote{By PRIMEUMATON at the first point, thy origin is sealed. By ANAPHAXETON at the second point, thy form is bound. By TETRAGRAMMATON at the third point, thy will is subject. Within this triangle thou art constrained. Thou canst not flee. Thou canst not hide. Thou must answer to the names of power inscribed herein.}
\end{quotation}
The constraint does not expel—it holds. Once the entity is trapped, expulsion protocols (Part III) can be applied with greater safety and efficacy.
\subsection{The Secret Seal of Solomon}
This seal appears in the Lemegeton as the mechanism by which spirits were bound into brass vessels \cite{peterson2001}. Its power is already generalized—it compels submission regardless of the entity's individual nature or rank.
The seal consists of:
\begin{itemize}
	\item Concentric circles forming a boundary
	\item A central square containing divine names (typically TETRAGRAMMATON and ADONAI)
	\item Hebrew letters arranged around the perimeter
	\item Geometric patterns that create a "lock" requiring specific knowledge to open
\end{itemize}
\textbf{Practical Applications:}
\begin{itemize}
	\item \textit{Threshold Installation:} Place beneath doorways or windows to trap entities attempting entry. The seal does not prevent passage—it captures what crosses.
	\item \textit{Container Lid:} Inscribe or affix to the lid of any vessel used for containment. Once sealed, the entity cannot escape without the seal being broken by external authority.
	\item \textit{Surface Inscription:} Carve into floors, walls, or objects where persistent disturbance is suspected. The seal binds what is present to that location.
\end{itemize}
\textbf{Activation:}
Unlike the Circle, which activates through spoken formula, the Secret Seal functions through physical closure. Once an entity is within its boundary, physically sealing the container or covering the inscription activates the binding. The seal must then be maintained—any break in the physical integrity compromises the trap.
\subsection{The Brass Vessel: Physical Containment}
The Lemegeton describes brass vessels for containing evoked spirits. For defensive work, the vessel serves as a portable trap—useful when an entity is bound to a specific object or when mobile containment is required.
\textbf{Construction:}
\begin{itemize}
	\item \textit{Body:} Brass (copper-zinc alloy). Brass conducts spiritual force while remaining stable. Minimum size: large enough to contain the bound object if applicable, or at least 6 inches in diameter for formless entities.
	\item \textit{Lid:} Lead. Lead restricts and binds (associated with Saturn—limitation, endings). The Secret Seal of Solomon must be impressed, engraved, or affixed to the underside of the lid.
	\item \textit{Base:} Iron plate beneath the vessel. Iron disrupts spiritual cohesion and prevents the entity from "sinking" through the bottom.
	\item \textit{Interior contents (optional but recommended):}
	\begin{itemize}
		\item Mercury in sealed glass vial (fluidity bound—prevents shapeshifting)
		\item Graveyard earth (association with the dead—grounds formless entities)
		\item Salt (purification and binding)
		\item Sulfur (traditional binding agent in alchemical work)
	\end{itemize}
\end{itemize}
\textbf{Usage Procedure:}
\begin{enumerate}
	\item If the entity is bound to an object: place the object inside the vessel
	\item If the entity is formless: use the Triangle of Art to constrain it to a specific location, then position the open vessel at the center of the triangle
	\item Speak the binding: \enquote{By the authority of Solomon who bound thy kind, by the Secret Seal impressed upon this vessel, by the names of power that compel thee—ENTER this vessel and be bound therein.}
	\item Seal the vessel immediately with the lead lid (Secret Seal facing inward)
	\item Bind the vessel with iron wire or chain, wrapping at least seven times
	\item Store in a location away from human habitation, preferably buried or encased in concrete
\end{enumerate}
\textbf{Warning:}
Once sealed, the vessel should not be opened except under controlled conditions with full protective protocols in place. The binding does not destroy the entity—it contains it. Breaking the seal releases what is inside.
\subsection{When to Use Each Method}
\begin{xltabular}{\textwidth}{|l|X|X|}
	\hline
	\textbf{Method} & \textbf{Use Case} & \textbf{Limitation} \\ \hline
	\endfirsthead
	\hline
	\textbf{Method} & \textbf{Use Case} & \textbf{Limitation} \\ \hline
	\endhead
	Triangle & Localized disturbance; entity present but not yet expelled & Temporary. Must be followed by expulsion or vessel containment. \\ \hline
	Secret Seal & Threshold protection; binding entity to specific location & Stationary. Only works where physically installed. \\ \hline
	Brass Vessel & Portable containment; object-bound entities; when expulsion is not possible & Permanent containment required. Must be maintained. \\ \hline
	\caption{Entrapment Methods}
\end{xltabular}
The Triangle isolates. The Seal binds. The Vessel contains. In practice, these methods often work in sequence: Triangle constrains, Seal reinforces, Vessel provides final containment if expulsion fails.
\subsection{Combining Entrapment with Expulsion}
Entrapment is not an end state—it is a control measure. Once the entity is contained within the Triangle or Vessel, expulsion protocols (Part III) can be applied with reduced risk. The entity cannot flee, cannot hide, and is forced to respond to the escalating conjurations.
This is the proper sequence for embedded or persistent intrusions:
\begin{enumerate}
	\item Constrain with Triangle
	\item Apply First Conjuration (command)
	\item If unsuccessful, reinforce with Secret Seal
	\item Apply Second Conjuration (threat)
	\item If still unsuccessful, transfer to Vessel
	\item Apply Third Conjuration (curse) with Babylonian or Catholic formulas as escalation
\end{enumerate}
Containment creates the conditions for effective expulsion. Without it, the entity may simply withdraw temporarily and return later.
