\section{ANCIENT HEBREW FORMULAS}
\sectionsubtitle{Second Temple \& Intertestamental Rites}

\subsection{The Zechariah Rebuke}
Recorded in the Prophet Zechariah \cite[3:2]{zechariah}, this is the foundational formula for Jewish and Christian expulsions. Unlike later rituals which address the spirit directly, this formula calls upon the Divine Prosecutor (The Angel of the LORD) to issue the verdict.

\rubric{The Formula}
\textit{\small To be declared with authority, visualizing the spirit standing before the Divine Court.}

\begingroup
\centering
% LATIN (Vulgate)
{\latinfont\Large
	Et dixit Dóminus ad Sáthan: \par
	Íncrepet Dóminus in te, Sáthan! \par
	Et íncrepet Dóminus in te, \par
	qui elégit Jerúsalem! \par
	Nonne hic torris est erútus de igne? \par
}

\vspace{1em}

% ENGLISH
{\itshape\small
	And the Lord said to Satan:
	The Lord rebuke thee, O Satan!
	And the Lord that chose Jerusalem rebuke thee!
	Is not this a brand plucked out of the fire? \par}
\endgroup

\subsection{The Fumigation of Tobit}
From the \textit{Book of Tobit} \cite[8:2--3]{tobit}. This rite is unique in that it uses a physical fumigation to drive away the demon Asmodeus. It is technically a rite of \textit{\gls{insufflation}}.

\inlineimage{tobias_woodcut.jpg}{Tobias extracting the liver}{tobias_fish}{koberger_bible}

\rubric{The Materials}
\begin{itemize}
	\item The heart and liver of a fish.
	\item Burning coals or incense.
\end{itemize}

\rubric{The Instruction}
\textit{\small "Then Tobias took the live coal, and laid the heart and liver of the fish on it, so as to make a smoke."}

\rubric{The Prayer of Deliverance}
\textit{\small Spoken after the fumigation is complete.}

\begingroup
\centering
% LATIN
{\latinfont\Large
	Benedíctus es, Deus patrum nostrórum, \par
	et benedíctum nomen tuum \par
	in ómnia saécula saeculórum: \par
	benedícant te caeli, \par
	et omnis creatúra tua. \par
}

\vspace{1em}

% ENGLISH
{\itshape\small
	Blessed art Thou, O God of our fathers,
	and blessed be Thy name forever and ever:
	let the heavens bless Thee,
	and all Thy creation. \par}
\endgroup

\newpage

\subsection{The Qumran Incantation}
Recovered from the Dead Sea Scrolls \cite{dss_11q11}, this text is one of the few surviving examples of a specialized "Psalm of Exorcism" used by the Essene community. It is attributed to David but functions as a direct interrogation of the entity.

\rubric{The Interrogation}

\begingroup
\centering
{\latinfont\large
	% Reconstruction of the Hebrew/Aramaic sense in English for clarity
	Who are you, \par
	you who were born from man \par
	and the seed of the holy ones? \par
	Your face is only an illusion, \par
	and your horns just a dream. \par
	Darkness you are, not light; \par
	Injustice, not righteousness. \par
}

\vspace{1em}
\rubriccross
\vspace{1em}

{\latinfont\large
	The Chief of the Army of YHWH \par
	will bind you in the deepest Sheol, \par
	and will close the bronze gates \par
	through which no light penetrates. \par
	The sun will not shine for you, \par
	that rises for the righteous! \par
}
\endgroup

\flourish

\subsection{The Rite of Eleazar}
The historian Josephus \cite[Antiquities 8.46--49]{josephus_antiquities} describes witnessing a Jewish exorcist named Eleazar performing this rite in the presence of the Emperor Vespasian. It relies on the "Solomonic Method" involving a ring and a specific root.

% Using your existing Solomon image since it fits the 'Ring/Seal' context perfectly
\inlineimage{solomon_hexagram.png}{The Seal of Solomon}{eleazar_seal}{josephus_antiquities}

\rubric{The Materials}
\begin{itemize}
	\item A ring containing the Root of Baaras (or Solomon's Root).
	\item A basin of water (set at a distance).
\end{itemize}

\rubric{The Procedure}
\begin{enumerate}
	\item \textbf{Extraction:} Apply the ring to the nostrils of the possessed. Draw the demon out through the scent of the root.
	\item \textbf{Abjuration:} As the subject falls, adjure the demon to return no more, reciting the \textit{Incantations of Solomon}.
	\item \textbf{The Proof:} Command the departing spirit to overturn the basin of water, as a physical sign of its exit.
\end{enumerate}

\rubric{The Solomonic Adjuration (Reconstruction)}

\begingroup
\centering
{\latinfont\Large
	In nómine Dei Sabaoth, \par
	per Sigíllum Salomónis, \par
	fuge et noli redíre! \par
}

\vspace{0.5em}

{\itshape\small
	In the name of the Lord of Hosts,
	by the Seal of Solomon,
	flee and do not return! \par}
\endgroup