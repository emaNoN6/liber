	\section{PRONUNCIATION GUIDE}
	\subsection{Sumero-Akkadian Phonetics}
	The following diacritical marks are used in the transcription of the Fifty Names to ensure correct pronunciation of the divine vibrations.
	
	\begin{xltabular}{\textwidth}{|c|l|X|}
		\hline
		\textbf{Symbol} & \textbf{Sound} & \textbf{Example} \\ \hline
		\endhead
		\v{s} & \textbf{sh} as in \textit{ship} & \textsc{\v{S}azu} (pronounced \textit{Shah-zoo}) \\ \hline
		\d{h} & \textbf{ch} as in Scottish \textit{loch} & \textsc{Asallu\d{h}i} (pronounced \textit{Ah-sall-oo-khee}) \\ \hline
		\d{t} & Emphatic \textbf{t}, hard stop & \textit{\d{T}uppu} (Tablet) \\ \hline		\d{s} & Emphatic \textbf{s} (ts sound) & \textit{Mu\d{s}a\d{s}ir} \\ \hline
		\={a}, \={e}, \={u} & Long vowels, held for double duration & \textit{E\={n}\={u}ma Eli\v{s}} \\ \hline
		\caption{Sumero-Akkadian Phonetics}
		\label{tab:pronunciation}
	\end{xltabular}
	
	\subsection[Buddhist chant thing]{U\d{s}\d{n}\={\i}\d{s}a Vijaya Dh\={a}ra\d{n}\={\i}} %{Uṣṇīṣa Vijaya Dhāraṇī}
	\begin{center}
		\begingroup
		\sanskritfont       % Switch to Sanskrit font
		\linespread{1.3}\selectfont % Give the tall letters room to breathe
		\Large              % Sanskrit looks better slightly larger than English
		
		% The text converted from IAST:
		नमो भगवते त्रैलोक्य प्रतिविशिष्टाय बुद्धाय भगवते तद्यथा ॐ विशोधय विशोधय असमसम समन्त अवभास स्फरण गति गहन स्वभाव विशुद्धे अभिषिञ्चतु मां सुगत वर वचन अमृत अभिषेके महामन्त्र पाने आहर आहर आयुः सन्धारणि शोधय शोधय गगन विशुद्धे उष्णीष विजय विशुद्धे सहस्ररश्मि सञ्चोदिते सर्व तथागत अवलोकन षट्पारमिता परिपूरणि सर्व तथागत मति दशभूमि प्रतिष्ठित सर्व तथागत हृदय अधिष्ठान अधिष्ठित महामुद्रे वज्रकाय सहरण विशुद्धे सर्व आवरण अपाय दुर्गति परिविशुद्धे प्रतिनिवर्तय आयुः शुद्धे समय अधिष्ठिते मणि मणि महामणि तथाता भूत कोटि परिशुद्धे विस्फुट बुद्धि शुद्धे जय जय विजय विजय स्मर स्मर सर्व बुद्ध अधिष्ठित शुद्धे वज्रे वज्र गर्भे वज्रं भवतु मम शरीरं सर्व सत्त्वानां च काय परिविशुद्धे सर्व गति परिशुद्धे सर्व तथागताश्च मे सम आश्वासयन्तु सर्व तथागत सम आश्वास अधिष्ठिते बुध्य बुध्य विबुध्य विबुध्य बोधय बोधय विबोधय विबोधय समन्त परिशुद्धे सर्व तथागत हृदय अधिष्ठान अधिष्ठित महामुद्रे स्वाहा
		\endgroup
	\end{center}

